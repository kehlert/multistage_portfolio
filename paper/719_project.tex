\documentclass{amsart}

\usepackage{amssymb, amsfonts, hyperref}
\usepackage{enumerate}

\newtheorem{thm}{Theorem}[section]
\newtheorem{cor}[thm]{Corollary}
\newtheorem{prop}[thm]{Proposition}
\newtheorem{lem}[thm]{Lemma}

\theoremstyle{definition}
\newtheorem{defn}[thm]{Definition}
\newtheorem{defns}[thm]{Definitions}
\newtheorem{con}[thm]{Construction}
\newtheorem{exmp}[thm]{Example}
\newtheorem{exmps}[thm]{Examples}
\newtheorem{notn}[thm]{Notation}
\newtheorem{notns}[thm]{Notations}
\newtheorem{addm}[thm]{Addendum}
\newtheorem{exer}[thm]{Exercise}

\theoremstyle{remark}
\newtheorem{rem}[thm]{Remark}
\newtheorem{rems}[thm]{Remarks}
\newtheorem{warn}[thm]{Warning}
\newtheorem{sch}[thm]{Scholium}

\bibliographystyle{plain}

%--------Meta Data: Fill in your info------
\title{Optimizing Multistage Portfolio Returns with Risk Constraints}

\author{Kurt Ehlert}

\date{\today}

\begin{document}

\begin{abstract}

TODO here goes the abstract

\end{abstract}

\maketitle

\tableofcontents

\section{model of market, why use geometric BM and risk free thing}

Write SDE of geometric BM, and explain why its different intuitively from regular BM. Put in reasons and also a picture of geometric BM trajectories vs regular BM trajectories. Include references on why to use geometric BM.\cite{hull}

Solve 1d geometric BM. Solve d dim. geometric BM with correlations. Include some plots of that. Also include equation and plot of returns when money is distributed equally between all assets.

\section{Markowitz's model}

reference those slides online and markowitz's paper. Setup the model. Solve it with Lagrange. When can't it be solved?

\begin{defn}  This is how to define a definition.
\end{defn}

And for a theorem and its proof you would type:

\begin{thm}
This is the statement of a theorem.
\end{thm}
\begin{proof}
And this shows that the statement is correct.
\end{proof}

Note that the numbering is taken care of automatically, and that we've predefined a bunch of these sorts of environments to take care lemmas, corollaries and such in the header. 

Another useful kind of enviroment is the equation environment.  Equations
get numbered in sequence with statements, as for example

\begin{equation}  e = mc^2
\end{equation}

Note if you do not want a numbered equation, you can use the
environment ``equation*''
 like so:

\begin{equation*}
e=mc^2
\end{equation*}

There are plenty of other equation-type enviroments that allow you to
align several equations and such. The AMS's guide is a
good place to start with these.

You can also typeset math directly in a paragraph by placing it within
dollar signs.  This is called ``math mode.''  For example: Let $e$
be energy, $m$ be momentum and $c$ be the speed of light.  Then
Einstein's famous equation says that $e=mc^2$.  This is useful, but
remember that it is harder to read inline math than displayed math. 

Remember that letters get put in a different font in math mode, so
whenever you are referencing a mathematical object you should always
put it in dollar signs.  For example, $f$ is a function, but f is just
a random letter.

Both and  have good lists of other symbols you can use in math mode.
These include greek letters ($\alpha, \beta, \Gamma, \Delta$),
operators ($\otimes, +, \sum$) and much more ($\leq, \diamond$).


\section{V@R, CV@R}

If you want to draw diagrams, you should use xypic.  It's actually
much easier than it looks, and we've already included it in the header
above.  Here is an example. 

\section{multistage}

\subsection*{Acknowledgments}  I downloaded and then modified the \LaTeX template at \url{www.math.uchicago.edu/~may/VIGRE/VIGRE2011/TEMPLATE.tex}

\bibliography{719_project}
\bibliographystyle{ieeetr}

\end{document}

